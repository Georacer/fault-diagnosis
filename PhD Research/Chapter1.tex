\chapter{Why Do UAVs Need FDD Systems}

Aerial robotic systems, also known as Unmanned Aerial Vehicles (UAVs) are becoming increasingly popular during the past decade. After seminal research on military applications ran its lifespan, smaller and less expensive systems found their way to research institutes globally. After the imperative and necessary control problems where solved and structural designs where streamlined, UAVs finally found their way to the consumer at the end of the 2000's, as a natural extension of the radio-controlled aircraft market.

The public quickly embraced this new technology and as of 2014, consumer Unmanned Aerial Systems (UASs) are widely used for aerophotography, ground surveys and recreation. However, the fact that UAVs are able to pilot themselves using well-established control algorithms for the nominal case, does not mean that they are able to remain operational in case of faults. Indeed, amateur UAV operators rarely perform Standard Operational Procedures, pre-flight checks or pre-emptive maintenance. As a result, the sight of a UAV falling out of the sky is all too common.

One would argue that operators should be the ones responsible for correct system practices, situation assessment and integrity checks. However, every time a technology gains wide acceptance to the public, eventually the task of safety assurance falls onto the manufacturer. As a comparison, consider how common and indispensable safety systems such as the ABS, ESP and vehicle diagnostics are in the automotive industry. If consumer UAVs are to become a part of our everyday lives, have a respectable degree of reliability and conquer the skies, safety standards must be built at the system level. This is the objective of the discipline of Fault Detection and Diagnosis (FDD)

Let us loosely describe Fault Detection and Diagnosis here and provide a more rigorous, mathematical definition later. Faults are events that throw a system out of its nominal mode of operation. Most of the times, faults, if allowed to go unchecked, will eventually result to system failures, which is an unwanted situation. Hence, it is of our interest to detect faults at their birth and take counter-measures against them, in order to ensure system health and if possible, maintain its operational status. Fault Detection involves setting up mathematical and physical structures that are able to detect when a fault occurs in our system. Afterwards, Fault Isolation should be performed, in order to identify the system component which is under fault. This is an important piece of information, since it allows us to target the source of a fault and is a necessary prerequisite for reconfigurable, fault-tolerant control (FTC). Finally, Fault Detection methods allow us to assess the magnitude of the fault, be it the drift of a physical parameter value, the flow of a leak or the strength of a disturbance. Oftentimes, Isolation and Detection are mutually referred to as Diagnosis.

