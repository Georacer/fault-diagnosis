\chapter{Other Future Directions}

\section{Residual Analysis}
Using formal ways to tackle large system complexity was shown to be beneficial for the purposes of Fault Detection and Diagnosis. In this text, the analysis ended with the production of residuals, which contain information on potential system faults. Inevitably, in order to extract final conclusions on system faults, these residuals have to be analyzed to recognize the appearance of fault signatures inside them. A plethora of methods is employed to achieve this goal, ranging from simple thresholding up to statistical tests and pattern recognition.

\section{Separating Faults from Disturbances}
Under the right system model, faults and disturbances can be handled in a similar fashion, and in the same ways that faults are detected and isolated and identified, so can disturbances be. This can prove beneficial for assessing the intensity of environmental disturbances.

\section{Hybrid Systems}
Hybrid systems have recently received a lot of attention in FDD and reconfigurable control applicaitons. Indeed, they offer an intuitive and formal way to handle events in the system, introduce temporal causality and define different system states, which in turn need different handling, in terms of monitoring, control and mission planning. Discrete Event Systems (DES) used along with the results from FMEA can also be used to back-track fault apparitions in time and have better estimates on fault isolation.

\section{Fault Identification and Reconfigurable/Restructurable Control}
If component failures can be precisely isolated, then the task of on-line system identification can be rendered a lot easier. If the system structure has not changed, then on-line identification algorithms can be focused on specific subsets of input/output signals to detect parameter changes with much greater accuracy. Thus adaptive controllers have an easier task to re-tune their gains for effective control. If structural changes are indeed detected, then this information can be used to perform re-structurable control.

\section{Mission Planning}
By having a constantly updated system structure/model, its limits and capabilities can be compared against mission goals to assess their feasibility in real-time. This not only adds an extra layer of safety for the UAV, since it enables it to abort un-achievable missions, but it may also enable it to re-specify its mission automatically in order to complete as much of the original mission as possible while being able to return to base at the same time.