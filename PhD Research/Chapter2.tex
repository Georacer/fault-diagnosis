\chapter{Historical FDD Overview}

Fault Detection and Diagnosis is a subject of research since the end of the 70's, originally intended for application in the petrochemical industry, which was a very innovative, wealthy and large-scale sector at that time. Employing mainly state and parameter estimation techniques, which where very costly computationaly-wise at the time, FDD was barely applicable to the slow-evolving chemical processes.

During the 90's, when adaptive control was a new topic of interest, FDD was re-visited to aid in building fault-tolerant control systems. Especially in the airline industry, which at that time was both booming and employing in-flight computers with system-wide span, efforts where made to design systems which would automatically intervene to the pilot's actions, to detect some basic faults and avert potential (and potentially very expensive and life-threatening) accidents. However, the costly experimental iterations, as well as the inability to guarantee the correct operation of the fault-tolerant control system, which would in turn jeopardize human lives, prevented FDD and FTC from reaching their full potential in airborne systems.

UAVs provide the low-cost and disposable test platform which is needed to verify the efficiency of FDD methods and are getting increasing interest with every successful research milestone.